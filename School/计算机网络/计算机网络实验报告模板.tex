\documentclass[12pt,a4paper]{article}
\usepackage[UTF8]{ctex}
\usepackage{geometry}
\usepackage{fancyhdr}
\usepackage{booktabs}
\usepackage{array}
\usepackage{xcolor}
\usepackage{tikz}
\usepackage{enumitem}
\usepackage{setspace}
\usepackage{multirow}

% 页面设置
\geometry{
    left=2.5cm,
    right=2.5cm,
    top=2.5cm,
    bottom=2.5cm
}

% 设置行距
\onehalfspacing

% 自定义命令
\newcommand{\fillblank}[2][3cm]{\underline{\makebox[#1][l]{#2}}}

% 标题样式
\title{\textbf{\Huge 计算机网络实验报告\#}}
\author{}
\date{}

% 设置表格列宽
\newcolumntype{L}[1]{>{\raggedright\arraybackslash}p{#1}}
\newcolumntype{C}[1]{>{\centering\arraybackslash}p{#1}}

\begin{document}

% 第一页:封面
\maketitle
\thispagestyle{empty}

\vspace{1cm}

\begin{center}
\begin{tabular}{ll}
\textbf{课程名称:} & \fillblank[6cm]{} \\
\\
\textbf{班级:} & \fillblank[6cm]{} \\
\\
\textbf{实验日期:} & \fillblank[6cm]{} \\
\\
\textbf{姓名:} & \fillblank[6cm]{} \\
\\
\textbf{学号:} & \fillblank[6cm]{} \\
\\
\textbf{实验名称:} & \fillblank[6cm]{} \\
\end{tabular}
\end{center}

\vspace{1.5cm}

\noindent\textbf{实验目的及要求:}\\[0.3cm]
\fillblank[16cm]{}\\[0.5cm]
\fillblank[16cm]{}\\[0.5cm]
\fillblank[16cm]{}

\vspace{1cm}

\noindent\textbf{实验环境:}\\[0.3cm]
\fillblank[16cm]{}\\[0.5cm]
\fillblank[16cm]{}

\vspace{1cm}

\noindent\textbf{实验内容:}\\[0.3cm]
\fillblank[16cm]{}\\[0.5cm]
\fillblank[16cm]{}\\[0.5cm]
\fillblank[16cm]{}

\newpage

% 第二页:实验步骤
\section*{实验步骤}

\vspace{0.5cm}

% 这里可以添加实验步骤的内容
% 可以使用 enumerate 或 itemize 环境

\begin{enumerate}[leftmargin=2em, itemsep=1em]
    \item 
    
    \item 
    
    \item 
    
    \item 
\end{enumerate}

\vspace{2cm}

% 可以添加更多内容区域
\vspace{10cm}

\newpage

% 第三页:关键问题及分析、总结
\section*{关键问题及分析}

\vspace{0.5cm}

% 关键问题及分析内容
\vspace{12cm}

\section*{总结}

\vspace{0.5cm}

% 总结内容
\vspace{12cm}

\newpage

% 第四页:实验成绩评定表
\section*{实验成绩评定表}

\vspace{1cm}

\begin{center}
\renewcommand{\arraystretch}{1.5}
\begin{tabular}{|L{10cm}|C{2cm}|C{2cm}|}
\hline
\centering\textbf{评价内容} & \textbf{权重} & \textbf{得分} \\
\hline
\multicolumn{3}{|l|}{\textbf{验收}} \\
\hline
实验原理是否理解;程序能否运行;实验结果是否正确;任务是否全部完成。 & 0.5 & \\
\hline
\multicolumn{3}{|l|}{\textbf{实验报告}} \\
\hline
1. 报告格式是否规范,语言使用是否规范,行文是否流畅,是否图文并茂; & 0.2 & \\
\hline
2. 实验原理、实验步骤描述是否正确、详实;程序流程图是否规范,代码实现是否正确;实验数据记录是否完整,实验结果是否正确;实验结果的分析、对比是否充分; & 0.2 & \\
\hline
3. 实验体会是否正确,是否提出了自己独到见解。 & 0.1 & \\
\hline
\textbf{合计} & \textbf{1.0} & \\
\hline
\end{tabular}
\end{center}

\end{document}

